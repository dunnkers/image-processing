\documentclass{article}
\usepackage{styles}

\title{Image Processing\\
    Lab 2}
\author{Kevin Gevers (s25595987) \\ Jeroen Overschie (s2995697)}
\date{\today}

\begin{document}

\maketitle

Note, that all used source code can be found attached next to this report's pdf file. Its structure should be self-explanatory; all requested functions are named accordingly and any extra functions are explained in the report. Note that (almost) every function has a corresponding test script, which is named just like its function, but with a suffix '\_test'. Where possible, we followed the terminology from the book \citep{gonzalez2008digital} for variable naming.

\section*{Exercise 1}

\section*{Exercise 2}
\subsection*{(a)} sdsd:

\[
\begin{aligned}
h(r_k)=n_k & \text{ for $k=0,1,2,...,L-1$}
\end{aligned}
\]
d
\[
s=T(r)=(L-1)\int_{0}^{1} p_r(w)dw
\]
\begin{figure}[ht]
    \centering
    \includesvg[width=\textwidth]{Assignment_1/output_plots/blurrymoon_all_histograms.svg}
    \caption{Example of the result of applying our histogram function, \textsc{IPhistogram} versus Matlab's built-in one.}
    \label{fig:histogram}
\end{figure}

\ding{118} Implementation: Listing~\ref{code:IPbhpf}.
\subsection*{(b)} sdsd:

\ding{118} Implementation: Listing~\ref{code:IPftfilter}.
\subsection*{(c)} sdsd:

\bibliographystyle{plain}
\typeout{}
\bibliography{Assignment_1}

\appendix
\section{Code}
\subsection{Exercise 1}
\subsection{Exercise 2}
\subsubsection{Exercise 2 (a)}
\lstinputlisting[caption={IPbhpf.m: Butterworth High-Pass Filter (BHPF).}, label={code:IPbhpf}]{Assignment_2/IPbhpf.m}
\subsubsection{Exercise 2 (b)}
\lstinputlisting[caption={IPftfilter.m: Convolute a filter in the frequency domain.}, label={code:IPftfilter}]{Assignment_2/IPftfilter.m}
\subsection{Exercise 3}
\end{document}
